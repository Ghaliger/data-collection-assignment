\documentclass[10pt,letterpaper]{article}
\begin{document}
\title{ THE INVENTION OF THE BIOMATRIC NATIONAL ID IN UGANDA.}
\author{by NABIMANYA LYNN  \\ 216008283 \\  16/U/8105/EVE}
\maketitle
\section{INTRODUCTION }
Perhaps one of the most revolutionary things to happen in Uganda in the last ten years would be the creation of a biometric National Identification Register in order to strengthen citizen identity management, national security and for the social economic development of the country.\\
 Though the idea was conceived in 2005, it was actualized after 2014 after Gen. Aronda Nyakairima, the former chief of defense forces, was appointed Minister of Internal Affairs / National Coordinator-National ID Project. The job was then handed over to National Identification and Registration Authority (NIRA), whose principle responsibility is the establishment and maintenance of the National Identification Register

\section{AIMS OF THE PROJECT; }
The main aims for creating biometric National ids included among others the registration of all citizens of Uganda; the registration of non-citizens of Uganda who are lawfully residents in Uganda; the registration of births, deaths and adoption orders; assignment of a unique National Identification Number (NIN) to every person registered; issuance of National Identification Cards and Aliens’ Identification Cards and also provide access and use of information contained in the National Identification Register in accordance with Regulations issued by the Board after consultation with the Minister.
\section{STRATEGIC OBJECTIVES}
1.   To identify, register and issue national identification numbers to citizen of Uganda.\\
2.  To issue National Identity cards to all Citizens of Uganda of 18 years and above.\\
3.    To identify and issue cards to foreign residents.\\
4.  To issue secure identification cards that enable Ugandans engage in economic and social-political activities.\\	
5.  To create a platform for integration with other databases of other agencies for ease of data sharing and effective service delivery.\\


\section{What are the major benefits of the National Identity Card?}
 The National Identity Card will;

a)	Facilitate the delivery of national development based on reliable and verifiable data:\\
i. Enable Government plan properly and provide easy access to good social facilities and services\\
ii. Benefiting from the National Health System and school capitation grants based on verifiable identities and data at the local level\\
iii. Facilitate more transparent and trustworthy business transactions\\
iv. Help keep crime low in our communities with quick and reliable identification of criminals\\
v. Guarantee the unique association “one-document/ one-identity” in the delivery of services, e.g., driver licensing, passports, visas.
vi. Help secure properties, title deeds and assets\\
b) Facilitate Regional and International integration based on verifiable and reliable data.\\
i. Support for regional integration (protocols for movement across the region)\\
ii. Conform to International obligations:\\

\section{REQUIREMENTS FFOR THE DATA COLLECTION}
All applicants are required to fill the Enrollment Form with the following information.\\
i.	Full name of the applicant.\\
ii. Residential address.\\
iii. Date of birth.\\
iv. Place of birth.\\
v. Indigenous community/tribe to which the applicant belongs.\\
vi. Place of origin.\\
vii. Occupation/profession.\\
ii.	A picture to identify the applicant.\\
iii.	Current location and GPS.\\
iv.	Spouse’s name.\\
v.	Parents name, nationally, tribe, and clan.\\

\section{CONCLUSION}
This research is appliedl, because we are trying to  solve general problems of  national  insecurity by strengthening citizen identity management, using biometric National Id. 
All in all, the project was and still is going on successfully, and in my opinion, it makes it seem like Uganda is finally catching up with the world.




\end{document}